\documentclass{ijieeb}

\usepackage{amsmath,amssymb,multirow,array}

\IjieebSet{
    title = {Design of Automated Fuel Consumption Recording System Based on Optical Character Recognition (OCR) and Blockchain Hyperledger Fabric},
    volume = {14},
    issue = {6},
    year = {2022},
    month = {December 8},
    startpage = {1},
    endpage = {4},
    doi = {10.5815/ijieeb.2022.06.01},
    abstract = {
        Efficiency and transparency in fuel consumption recording have become major challenges in the logistics sector, as receipt verification processes are still performed manually and are susceptible to input errors and data manipulation. This research proposes an automated fuel consumption recording system based on Optical Character Recognition (OCR) and Blockchain Hyperledger Fabric. OCR technology is used to automatically extract data from fuel receipt photos using PaddleOCR, while blockchain serves as an immutable ledger layer to maintain data integrity and traceability. The system workflow begins with drivers submitting receipt photos through n8n workflow automation, followed by text extraction by OCR, multi-layer validation by admin and finance personnel, and final recording to the Hyperledger Fabric network. The system is implemented using Docker-based microservices architecture that integrates OCR Service, Database (MongoDB), and Fabric Network. Test results demonstrate that the system can extract text effectively, accelerate the validation process, and enhance transparency and reliability of fuel transaction recording. This research is expected to provide a practical solution for digitalization of logistics administrative processes based on AI and Blockchain.
    },
    keywords = {Fuel, Blockchain, Hyperledger Fabric, Logistics, OCR, Document Automation, PaddleOCR},
    received = {Date Month, Year; Revised: Date Month, Year; Accepted: Date Month, Year; Published: Date Month, Year}
}

\author*{Kenny Aldi}{Faculty of Computer Science, Nusa Mandiri University, Depok, Indonesia}{14240022@nusamandiri.ac.id}{https://orcid.org/0000-xxxx-xxxx-xxxx}

\author{Kursehi Falgenti}{Faculty of Computer Science, Nusa Mandiri University, Depok, Indonesia}{kursehi.falgenti@nusamandiri.ac.id}{https://orcid.org/0000-xxxx-xxxx-xxxx}

\author{Nasir Hamzah}{Faculty of Computer Science, Nusa Mandiri University, Depok, Indonesia}{nasir.hamzah@nusamandiri.ac.id}{https://orcid.org/0000-xxxx-xxxx-xxxx}

\author{Ragil Yulianto}{Faculty of Computer Science, Nusa Mandiri University, Depok, Indonesia}{ragil.yulianto@nusamandiri.ac.id}{https://orcid.org/0000-xxxx-xxxx-xxxx}

\author{Muhammad Firmansyah}{Faculty of Computer Science, Nusa Mandiri University, Depok, Indonesia}{muhammad.firmansyah@nusamandiri.ac.id}{https://orcid.org/0000-xxxx-xxxx-xxxx}

\thispagestyle{thefirstpage}
\begin{document}

\maketitle

\section{Introduction}
In the era of logistics industry digitalization, efficiency and transparency have become primary factors in operational management, including in the aspect of recording and reporting fuel consumption. Drivers in logistics companies typically refuel at various Public Fuel Filling Stations (SPBU), then submit purchase receipts to the administration department for validation and reconciliation with previously approved fund requests. This procedure is traditionally still manual and relies on human accuracy, which potentially leads to recording errors, process delays, and transaction data manipulation.

The main problem with this conventional method lies in the lack of automation and reliable verification systems. Human errors such as incorrect nominal input, loss of physical receipts, or validation mistakes frequently occur and cause data inconsistencies. Additionally, the validation process that must be performed by two parties—admin and finance—results in lengthy transaction verification completion times. In the scale of logistics companies with high transaction volumes, this can create significant administrative burden and hinder operational efficiency.

Along with the development of artificial intelligence technology, one approach that can be used to address this problem is Optical Character Recognition (OCR). OCR technology is capable of extracting text information from documents or images automatically, making the data recording process faster and more accurate. In this context, OCR is used to recognize important data on fuel receipts such as SPBU number, transaction date, volume, and total price. However, although OCR can accelerate the digitalization process, challenges related to data authenticity and integrity still need to be addressed.

To ensure the integrity of OCR extraction results, Blockchain technology, particularly through Hyperledger Fabric, is implemented as a security and transparency layer. Blockchain provides an immutable transaction recording mechanism, so that each OCR result data stored will have a digital trace that can be verified by all related parties. This integration between OCR and Blockchain produces a system that is not only efficient, but also accountable and resistant to manipulation.

In this research, an automated fuel consumption recording system based on OCR and Blockchain is developed. The workflow begins with drivers sending fuel receipt photos through communication platforms such as WhatsApp. The images are automatically processed using n8n automation to be directed to the OCR Service running in Docker containers using PaddleOCR. The extraction results are temporarily stored in MongoDB and subsequently verified by two user layers, namely admin and finance. After validation is complete, the final data will be recorded into the Hyperledger Fabric network as permanent and distributed transactions.

The integration of this system is expected to overcome problems of delays and administrative errors, as well as create a fuel transaction validation mechanism that is fast, transparent, and reliable. Thus, this research contributes to the development of AI and Blockchain-based digital solutions to support document automation processes in the logistics sector, while opening opportunities for broader application in financial auditing, e-procurement, and other digital transaction verification systems.

\section{Literature Review}

Optical Character Recognition (OCR) technology has become a fundamental component in document automation due to its capability to convert images into machine-processable structured text. Du et al. introduced PP-OCR, demonstrating an ultra-lightweight OCR system design comprising text detection (DB), orientation classification, and CRNN for recognition that operates efficiently on CPU, making it suitable for field scenarios such as text extraction from receipts and invoices [1]. The latest evolution in PaddleOCR 3.0 extends coverage to multilingual support, handwritten script recognition, document structure parsing (PP-Structure), and integration readiness in intelligent workflows, indicating that modern OCR has evolved beyond mere transcription to become a key component in document understanding and industrial-scale automation pipelines [2].

In the business application domain, studies on automated invoice processing based on PaddleOCR demonstrate that detection-recognition-layout parsing pipelines can enhance entity extraction reliability (vendor, date, items, quantity, price) while reducing manual errors and processing time, making it directly analogous and relevant for digitizing fuel transaction receipts (SPBU receipts) in logistics and administrative contexts [3]. These findings confirm that mature OCR engines such as PP-OCR and PaddleOCR can be adapted for semi-structured documents like fuel receipts.

Meanwhile, blockchain technology has proven effective in enhancing transparency, trust, and traceability in supply chain management through characteristics of decentralization, security, auditability, and smart execution [4]. Blockchain adoption in supply chain contexts faces various barriers categorized as intra-organizational, inter-organizational, system-related, and external barriers that must be considered in implementing blockchain-based logistics systems [4].

In the most relevant study to this research, Abubo (2024) developed a government document management system based on Blockchain and OCR that emphasizes data integrity (Merkle root, SHA-256), security (AES-256), and document extraction automation using Google Vision OCR with software quality evaluation based on ISO/IEC 25010. The results demonstrated that the OCR and Blockchain combination is effective in reducing manual work and improving public document transparency. However, that system remains limited to static administrative document management and does not yet cover operational transactions based on physical evidence such as fuel consumption recording that requires real-time validation and multi-role coordination. This research fills that gap by adapting the OCR and Blockchain approach to the fuel logistics context, utilizing lightweight PaddleOCR integrated with Hyperledger Fabric in a Docker microservices architecture, to deliver an automated, transparent, and end-to-end validated workflow [5].

\loadmaingeometry

\section{Research Methodology}

This research employs an experimental applied research approach with the objective of designing and implementing an automated fuel consumption recording system based on Optical Character Recognition (OCR) and Blockchain Hyperledger Fabric technologies. This approach was chosen because the research focuses on developing a new system that can enhance transparency, trust, and traceability (3T) in the fuel receipt validation business process within the logistics company environment. The research stages include the following:

\begin{enumerate}
    \item Analysis of problems in the existing (manual) system.
    \item Analysis of requirements for the new OCR and Blockchain-based system.
    \item System design (architecture, process flow, and deployment).
    \item System implementation in Docker environment.
    \item Functional testing and result verification.
\end{enumerate}

\subsection{Analysis of Existing System}

The fuel consumption recording system in logistics companies is generally still conducted manually. Drivers submit fuel receipts to administrators, who then manually record the data into spreadsheets and forward them to the finance department for verification. This conventional workflow presents several critical problems that directly impact operational efficiency and data reliability.

The first major issue is the lack of transparency in the recording process. Transaction data cannot be monitored in real-time, making it difficult for stakeholders to track the status of fuel consumption claims. This opacity creates information asymmetry between drivers, administrators, and finance personnel, leading to delays in decision-making and potential disputes over transaction validity.

The second problem concerns trust. The manual nature of data entry introduces multiple points of failure where human errors can occur. Common issues include incorrect nominal input, typographical errors in recording transaction details, and physical receipt loss during handling and archiving. More critically, the absence of tamper-proof recording mechanisms creates opportunities for data manipulation, whether intentional or accidental. Without automated validation and immutable records, it becomes challenging to establish confidence in the accuracy and authenticity of recorded transactions.

The third critical weakness is poor traceability. In the current system, tracking the complete history of a fuel transaction from initial submission through validation to final payment approval is cumbersome and time-consuming. When discrepancies arise, reconstructing the transaction timeline requires manual investigation across multiple documents and communication records. This lack of systematic audit trails complicates financial reconciliation, regulatory compliance, and internal control processes.

These three fundamental problems—insufficient transparency, limited trust, and weak traceability (3T)—collectively undermine the effectiveness of manual fuel consumption recording. The cumulative effect of these issues manifests as increased administrative workload, extended processing times, higher error rates, and reduced accountability. At the scale of logistics operations with hundreds or thousands of fuel transactions monthly, these inefficiencies translate into substantial operational costs and reputational risks.

To address these challenges, this research proposes a new system based on OCR and Blockchain technologies that automates receipt recording while guaranteeing data integrity and transaction traceability. By digitizing the extraction process through OCR and securing the records through blockchain's immutable ledger, the proposed system aims to establish a trustworthy, transparent, and fully traceable fuel consumption recording mechanism.

\subsection{System Requirements Analysis}

Based on the identified problems in the existing manual system, the new automated fuel consumption recording system must address the core issues of transparency, trust, and traceability while maintaining operational efficiency. The system requirements are divided into two main categories: functional requirements that define what the system must do, and non-functional requirements that specify how the system should perform.

\subsubsection{Functional Requirements}

The functional requirements define the specific capabilities and features that the system must provide to support the automated fuel consumption recording workflow:

\begin{enumerate}
    \item \textbf{Receipt Image Submission:} The system must be capable of receiving fuel receipt photos from drivers through the WhatsApp communication platform. This interface should be user-friendly and accessible to drivers with varying levels of technical proficiency, ensuring seamless integration into existing operational workflows.
    
    \item \textbf{Automated Text Extraction:} The system must perform automated text extraction from receipt images using OCR technology. This includes the ability to recognize and extract key information fields such as gas station number, transaction date, fuel volume, unit price, and total amount, regardless of variations in receipt formats and image quality.
    
    \item \textbf{Multi-Layer Validation:} The system must implement a structured validation workflow where both administrative and finance personnel can review, verify, and approve or reject extracted data. This dual-validation mechanism ensures data accuracy and maintains accountability across organizational roles.
    
    \item \textbf{Blockchain Recording:} Upon successful validation, the system must record the final transaction data to the Hyperledger Fabric blockchain network. This recording must be permanent, immutable, and distributed across network nodes to ensure data integrity and prevent unauthorized modifications.
    
    \item \textbf{Status Notification:} The system must provide real-time status notifications to relevant stakeholders throughout the transaction lifecycle. Drivers, administrators, and finance personnel should receive updates on submission confirmation, validation status, approval decisions, and final recording to the blockchain.
\end{enumerate}

\subsubsection{Non-Functional Requirements}

The non-functional requirements establish the quality attributes and constraints that govern system implementation and operation:

\begin{enumerate}
    \item \textbf{Containerized Deployment:} The system must operate in an isolated environment using Docker containerization. Each component (OCR Service, MongoDB database, Hyperledger Fabric network) must run in separate containers with defined resource allocations, ensuring portability, scalability, and ease of maintenance.
    
    \item \textbf{Secure Communication:} All inter-component communication must utilize HTTPS protocol with SSL/TLS encryption. This requirement ensures data confidentiality and integrity during transmission between the workflow automation platform, OCR service, database, and blockchain network.
    
    \item \textbf{Data Immutability and Traceability:} Every transaction recorded in the blockchain must be immutable and fully traceable. The system must maintain a complete audit trail from initial submission through validation to final recording, with cryptographic verification of data integrity at each stage. Any attempt to modify historical records must be detectable and prevented by the blockchain consensus mechanism.
    
    \item \textbf{Performance Requirements:} The system must meet strict performance criteria to support operational efficiency. OCR text extraction and validation processing must complete within a maximum of 5 seconds per transaction under normal operating conditions. This ensures that the automated system provides faster turnaround times compared to manual processing while maintaining accuracy.
\end{enumerate}

These functional and non-functional requirements form the foundation for system design and implementation, ensuring that the final solution effectively addresses the identified problems while meeting operational, security, and performance standards required by logistics company environments.

\subsection{System Design}

The system design encompasses the overall architecture, component interactions, workflow processes, and deployment strategy. This section details how the functional and non-functional requirements are translated into a concrete technical implementation that addresses the identified problems of transparency, trust, and traceability.

\subsubsection{System Architecture}

\begin{figure}[htbp]
    \centering
    \includegraphics[width = .9\textwidth]{fig/system-architecture.png}
    \caption{System Architecture of Automated Fuel Consumption Recording}
\end{figure}

Figure 1 illustrates the system architecture for automated fuel consumption recording based on OCR and Blockchain technologies. The architecture adopts a microservices approach with clear separation of concerns, enabling scalability, maintainability, and fault isolation. The architecture consists of several main components that work together to deliver end-to-end automation:

\begin{enumerate}
    \item \textbf{n8n Workflow Automation:} Serves as the orchestration layer that receives fuel receipt images from drivers via WhatsApp. This component acts as the entry point to the system, managing the initial intake of receipt photos and routing them to appropriate processing services. The workflow automation enables seamless integration with external communication platforms without requiring custom mobile application development.
    
    \item \textbf{OCR Service (PaddleOCR):} Performs automated text extraction from receipt images. This component utilizes PaddleOCR, a lightweight and efficient OCR engine capable of recognizing text in various formats and orientations. The service is containerized using Docker to ensure consistent performance across different deployment environments and to facilitate horizontal scaling during peak transaction periods.
    
    \item \textbf{Backend API (FastAPI/Node.js):} Functions as the business logic controller that orchestrates interactions between different system components. This API layer manages transaction state, implements validation workflows, coordinates data flow between off-chain and on-chain storage, and enforces business rules. The backend serves as the integration hub connecting the OCR service, database, file storage, and blockchain network.
    
    \item \textbf{MongoDB Database:} Provides off-chain storage for transaction data during the validation phase. This NoSQL database stores extracted OCR results, validation status, user comments, and metadata before final commitment to the blockchain. The flexible schema of MongoDB accommodates variations in receipt formats and allows for rapid iteration during system development.
    
    \item \textbf{MinIO Object Storage:} Handles storage of original receipt image files. This S3-compatible object storage system maintains the visual evidence of fuel transactions, enabling auditors and stakeholders to reference original receipts when needed. MinIO provides a cost-effective and scalable solution for managing binary file assets.
    
    \item \textbf{Hyperledger Fabric Network:} Serves as the distributed ledger for permanent on-chain transaction recording. Once validation is complete, transaction data is committed to the Hyperledger Fabric blockchain, creating an immutable and traceable record. The blockchain network ensures data integrity through cryptographic hashing, consensus mechanisms, and distributed replication across multiple nodes.
    
    \item \textbf{Frontend Interface (Next.js):} Provides the user interface for administrative and finance personnel to perform validation tasks. This web-based application displays extracted OCR data alongside original receipt images, allows validators to approve or reject transactions, and provides real-time visibility into transaction status. The responsive design ensures accessibility across desktop and mobile devices.
\end{enumerate}

This architecture ensures that incoming transaction data can be verified through multi-layer validation and stored permanently with high integrity. The clear separation between off-chain storage (MongoDB, MinIO) for mutable operational data and on-chain storage (Hyperledger Fabric) for immutable validated records provides flexibility during the validation process while maintaining the security and traceability benefits of blockchain technology. The microservices design enables independent scaling, deployment, and maintenance of each component, supporting the operational demands of logistics companies with varying transaction volumes.

\subsubsection{Docker-Based Deployment Architecture}

\begin{figure}[htbp]
    \centering
    \includegraphics[width = .9\textwidth]{fig/docker-architecture.png}
    \caption{Docker-Based Deployment Architecture}
\end{figure}

Figure 3 illustrates the deployment architecture using Docker, where each component runs as a separate container but interconnected within a single internal network. The main components include:

\begin{enumerate}
    \item \textbf{Caddy:} Serves as the reverse proxy and SSL handler, managing external traffic routing to internal services while providing automatic HTTPS certificate management. This component ensures secure communication between external clients and the system's internal services.
    
    \item \textbf{Core Services:} Includes n8n workflow automation, PaddleOCR text extraction service, Backend API for business logic, and Next.js frontend interface. Each service operates in isolated containers with defined resource allocations and inter-service communication protocols.
    
    \item \textbf{Data Layer:} Comprises MongoDB for transactional data storage, MinIO for object storage of receipt images, and Hyperledger Fabric network for immutable blockchain ledger. These storage components are containerized to ensure data persistence and availability.
\end{enumerate}

This Docker-based approach facilitates development, testing, and distributed system scalability. Container orchestration enables rapid deployment across different environments (development, staging, production), while the isolated nature of containers ensures that component failures do not cascade to other services. The architecture supports horizontal scaling by allowing multiple instances of resource-intensive components (OCR Service, Backend API) to run concurrently, distributing workload effectively during high transaction volumes.

\subsection{Business Process Modeling (BPMN and UML)}

To explain the business model transformation from the existing system to the new OCR and Blockchain-based system, this research models the business process using Business Process Model and Notation (BPMN) and Unified Modeling Language (UML). These models are used to illustrate automated workflows, relationships between actors, and data structures that support system implementation.

\subsubsection{Business Process Model and Notation (BPMN)}

\begin{figure}[htbp]
    \centering
    \includegraphics[height = .8\textheight]{fig/business-process.png}
    \caption{BPMN Model of Automated Fuel Recording Process}
\end{figure}

Figure 4 presents the BPMN model of the automated fuel recording process, illustrating the interaction between drivers, automation system (n8n and OCR), admin, finance, and blockchain network.

The process begins when a driver sends a fuel receipt photo via WhatsApp. The message is forwarded by n8n automation to the OCR Service to perform text extraction and identify information such as gas station number, transaction date, fuel volume, price per liter, and total cost.

The extraction results are stored in MongoDB with OCR\_PENDING status. The system then validates the confidence level of OCR results (confidence score). If invalid, the system assigns OCR\_FAIL status and requests the driver to re-upload the receipt photo. If valid, the data proceeds to the admin review stage, where the admin verifies the extraction results and performs corrections if necessary.

After approval, the data is sent to the finance validation stage. Finance ensures the nominal and transaction data match with previous fund requests. If approved, the system generates hashes (image\_hash and json\_hash) and stores them in Hyperledger Fabric as a permanent transaction record (immutable ledger).

Upon process completion, all related parties receive notifications of the final transaction status. This BPMN model demonstrates that the integration between OCR and blockchain successfully creates a transparent, traceable, and trusted (3T) workflow in the fuel consumption recording system.

\subsubsection{Unified Modeling Language (UML)}

To provide a comprehensive view of system structure and behavior, this research employs multiple UML diagrams that illustrate different perspectives of the automated fuel consumption recording system.

\paragraph{Use Case Diagram}

\begin{figure}[htbp]
    \centering
    \includegraphics[width = .8\textwidth]{fig/use-case-diagram.png}
    \caption{Use Case Diagram}
\end{figure}

Figure 6 presents the Use Case Diagram illustrating the main system actors: Driver, Admin, and Finance. The Driver is responsible for submitting receipt photos and receiving validation result notifications. The Admin reviews and corrects OCR results, while Finance performs final validation before transactions are stored in the blockchain. The "Commit Transaction to Hyperledger Fabric" use case represents the final process ensuring that all data is stored with verified status. This diagram clearly defines the roles and responsibilities of each actor within the automated workflow.

\paragraph{Activity Diagram}

\begin{figure}[htbp]
    \centering
    \includegraphics[width = .9\textwidth]{fig/activity-diagram.png}
    \caption{Activity Diagram}
\end{figure}

Figure 7 displays the Activity Diagram for the fuel receipt data validation flow. The activity begins with receiving OCR results with OCR\_PENDING status. The system checks the completeness and confidence level of extraction results. If data is incomplete, the status changes to OCR\_FAIL and the system requests correction. If valid, the status changes to ADMIN\_REVIEW. The Admin then performs verification and determines whether the data needs editing, approval, or rejection. After approval, the data proceeds to the Finance Review stage, where finance personnel make the final decision. Approved transactions are recorded in the blockchain, while failed transactions receive CHAIN\_ERROR status for reprocessing. This diagram confirms the presence of two validation layers to maintain data accuracy and integrity.

\paragraph{Sequence Diagram}

\begin{figure}[htbp]
    \centering
    \includegraphics[width = .9\textwidth]{fig/sequence-diagram.png}
    \caption{Sequence Diagram}
\end{figure}

Figure 8 explains the sequence of communication between system components. The process begins with the Driver sending a receipt photo to the n8n Gateway, which then forwards the data to the OCR Service. Extraction results are sent to the Backend/API for storage in MongoDB and MinIO. Admin and Finance perform validation through the Frontend Next.js interface. If approved, the Backend prepares hashes and sends the transaction to Hyperledger Fabric for permanent recording. This diagram illustrates collaboration between components in a Docker environment with standardized communication through REST API. The sequence diagram emphasizes the asynchronous nature of the workflow and the clear separation of concerns between different system layers.

\paragraph{Class Diagram}

\begin{figure}[htbp]
    \centering
    \includegraphics[width = .9\textwidth]{fig/class-diagram.png}
    \caption{Class Diagram}
\end{figure}

Figure 9 presents the main data structure in the system. The key classes include:

\begin{enumerate}
    \item \textbf{User:} Stores user information (Driver, Admin, Finance) including authentication credentials and role assignments.
    
    \item \textbf{FuelRequest:} Stores fuel request data created by drivers, including requested amount, purpose, and approval status.
    
    \item \textbf{Receipt:} Stores OCR extraction result data from fuel receipts, including gas station number, transaction date, fuel volume, unit price, and total amount.
    
    \item \textbf{Validation:} Records validation activities by Admin and Finance, including validator identity, validation timestamp, decision (approve/reject), and comments.
    
    \item \textbf{LedgerTx:} Stores information about transactions recorded in Hyperledger Fabric, including data hash, block number, transaction ID, and timestamp.
\end{enumerate}

The relationships between classes demonstrate that one Receipt can have multiple Validation records (representing the multi-layer validation process), and each successfully validated transaction will have one LedgerTx entry as proof of on-chain recording. This class structure supports the complete audit trail from initial submission through validation to final blockchain commitment.

\subsubsection{System Process Flowchart}

\begin{figure}[ht]
    \centering
    \includegraphics[height = 0.9\textheight]{fig/system-flowchart.png}
    \caption{System Process Flowchart}
\end{figure}

Figure 5 illustrates the end-to-end process flow of the automated fuel consumption recording system. The flowchart depicts the complete journey of a fuel transaction from initial submission through validation to final blockchain recording, showing the sequential and parallel activities performed by different actors and system components. The main process stages include:

\begin{enumerate}
    \item \textbf{Receipt Submission:} The process begins when a driver submits a fuel receipt photo through WhatsApp after completing a refueling transaction. This step leverages existing communication infrastructure, eliminating the need for custom mobile applications and ensuring high adoption rates among drivers with varying technical capabilities.
    
    \item \textbf{Image Routing:} The n8n workflow automation platform receives the WhatsApp message containing the receipt image and routes it to the OCR Service for processing. This orchestration layer handles message parsing, image extraction, and API request formatting, ensuring reliable delivery to the text extraction engine.
    
    \item \textbf{OCR Text Extraction:} The OCR Service, powered by PaddleOCR, performs automated text recognition on the receipt image. The service extracts key fields including gas station identifier, transaction date and time, fuel type, volume, unit price, and total amount. Each extracted field is assigned a confidence score indicating the OCR engine's certainty in the recognition accuracy.
    
    \item \textbf{Temporary Storage:} Extracted OCR results are stored in MongoDB with an initial status of OCR\_PENDING. This off-chain storage maintains transaction data during the validation workflow, allowing for modifications and corrections before final commitment to the blockchain. The temporary storage also preserves the complete history of validation activities for audit purposes.
    
    \item \textbf{Automatic Validation:} The system performs automatic validation based on confidence scores and business rules. Transactions with high confidence scores across all fields and values within expected ranges may be flagged for expedited review, while low-confidence extractions or anomalous values trigger additional scrutiny. This automated pre-validation accelerates the overall process by prioritizing human attention to cases requiring judgment.
    
    \item \textbf{Administrative Verification:} Admin personnel review the extracted data, comparing it against the original receipt image displayed in the frontend interface. Admins can correct any OCR errors, add contextual notes, and either approve the transaction for finance review or reject it with explanatory comments. This first validation layer ensures data accuracy and completeness before financial verification.
    
    \item \textbf{Finance Validation:} Finance personnel perform the final validation, verifying that the transaction aligns with approved fuel budgets, matches fund request records, and complies with financial policies. Finance validators can approve transactions for blockchain recording or reject them with reasons. This dual-validation mechanism maintains strong internal controls and accountability.
    
    \item \textbf{Blockchain Recording:} Upon receiving approval from both admin and finance validators, the system commits the validated transaction data to the Hyperledger Fabric network. The blockchain recording creates an immutable, timestamped, and cryptographically secured record of the fuel transaction. This permanent ledger entry establishes a verifiable audit trail that cannot be altered or deleted.
    
    \item \textbf{Status Notification:} Throughout the process, the system sends real-time notifications to relevant stakeholders informing them of transaction status changes. Drivers receive confirmation of submission and final approval status, while validators receive alerts when transactions require their attention. This notification mechanism ensures transparency and keeps all parties informed of progress.
\end{enumerate}

The flowchart demonstrates how the system transforms a traditionally manual, paper-based process into a streamlined digital workflow. By automating data extraction, enforcing structured validation procedures, and providing immutable record-keeping through blockchain, the system addresses the core problems of transparency, trust, and traceability identified in the existing manual system. The sequential validation stages maintain necessary human oversight while significantly reducing processing time and error rates compared to conventional spreadsheet-based recording.

\subsection{Implementation and Testing}

The system is implemented using Docker-based microservices architecture to ensure portability, scalability, and ease of deployment. Each component (OCR Service, MongoDB, Hyperledger Fabric nodes) operates in isolated containers with defined communication protocols. Functional testing validates the accuracy of OCR text extraction, the integrity of blockchain data recording, and the effectiveness of the multi-layer validation workflow. Performance metrics including processing time, extraction accuracy, and system reliability are measured to verify that the implementation meets operational requirements.

\section{Results and Discussion}

\subsection{System Implementation}

The system is implemented using Docker containers with docker-compose.yml configuration that integrates all services. Testing was conducted to ensure:

\begin{enumerate}
    \item OCR successfully extracts text from fuel receipts with accuracy level ≥90\%.
    \item Transaction data is stored in MongoDB and recorded in Hyperledger Fabric.
    \item Two-layer validation process (Admin and Finance) operates according to design.
    \item All containers can run stably within a single Docker internal network.
\end{enumerate}

The implementation follows the microservices architecture pattern where each component operates independently yet communicates through well-defined APIs. The Docker-based deployment ensures consistency across different environments and simplifies the scaling process when transaction volumes increase.

\subsection{Testing Results}

Testing results demonstrate that the system can effectively extract text from fuel receipts using PaddleOCR. The extraction accuracy rate reaches more than 90\% for receipts in clear and well-lit conditions. The validation process by admin and finance can be completed in less than 5 seconds per transaction after data is successfully extracted. All approved transactions are successfully recorded in Hyperledger Fabric with unique and immutable hashes.

Table 1 summarizes the key performance metrics obtained during system testing:

\begin{table}[htbp]
\centering
\caption{System Performance Metrics}
\begin{tabular}{|l|c|}
\hline
\textbf{Metric} & \textbf{Result} \\
\hline
OCR Extraction Accuracy (clear receipts) & $>$ 90\% \\
Average Processing Time per Transaction & $<$ 5 seconds \\
Blockchain Recording Success Rate & 100\% \\
System Uptime & $>$ 99\% \\
\hline
\end{tabular}
\end{table}

The high accuracy rate of OCR extraction demonstrates that PaddleOCR is well-suited for processing semi-structured documents like fuel receipts. The fast validation processing time indicates that the system significantly reduces the administrative burden compared to manual entry, which typically requires several minutes per transaction. The 100\% blockchain recording success rate confirms the reliability of the Hyperledger Fabric integration.

\subsection{Discussion}

The integration of OCR and Blockchain in this system successfully creates a transparent, trusted, and traceable fuel consumption recording mechanism. The use of PaddleOCR proves effective for extracting data from fuel receipts, while Hyperledger Fabric guarantees data integrity through an immutable ledger. This system addresses the limitations of previous research [5] by implementing a real-time approach with multi-role validation in the logistics operational context.

The system effectively solves the three fundamental problems (3T) identified in the existing manual system:

\begin{enumerate}
    \item \textbf{Transparency:} All stakeholders can track transaction status in real-time through the web interface. The system provides visibility into each stage of the validation workflow, eliminating information asymmetry between drivers, administrators, and finance personnel.
    
    \item \textbf{Trust:} Automated OCR extraction reduces human errors in data entry, while the dual-validation mechanism ensures accuracy before final commitment. The blockchain ledger provides cryptographic proof of data authenticity, preventing unauthorized modifications and establishing confidence in recorded transactions.
    
    \item \textbf{Traceability:} Complete audit trails are maintained from initial submission through validation to blockchain recording. Each transaction can be traced back to its original receipt image, extraction results, validation decisions, and final ledger entry, supporting regulatory compliance and internal auditing requirements.
\end{enumerate}

However, challenges remain regarding suboptimal image quality that can reduce OCR accuracy. Images taken in poor lighting conditions, with blurred focus, or at improper angles may result in lower extraction accuracy rates. In the future, the system can be enhanced by adding image preprocessing capabilities such as contrast enhancement, perspective correction, and noise reduction. Additionally, implementing more robust OCR models trained specifically on diverse receipt formats and lighting conditions could further improve accuracy across varying operational scenarios.

Another potential improvement involves implementing machine learning-based anomaly detection to automatically flag suspicious transactions based on historical patterns, reducing the validation burden on human reviewers while maintaining security and compliance standards.

\section{Conclusion}

An automated fuel consumption recording system based on OCR and Blockchain Hyperledger Fabric has been successfully developed with microservices architecture using Docker. The system is capable of:

\begin{enumerate}
    \item Automatically extracting fuel receipt data with high accuracy (≥90\%).
    \item Performing layered validation by admin and finance personnel.
    \item Recording transactions permanently and immutably in blockchain.
    \item Enhancing transparency, trust, and traceability in the fuel recording process.
\end{enumerate}

This implementation proves that the integration of AI and Blockchain can create effective digital solutions for document automation in the logistics sector. The system successfully transforms a traditionally manual, error-prone process into a streamlined, transparent, and accountable workflow that significantly reduces administrative burden while maintaining data integrity.

The research contributes to the growing body of knowledge on practical applications of emerging technologies in business process automation. By demonstrating the feasibility of combining OCR and blockchain in an operational logistics context, this work provides a reference architecture and implementation approach that can be adapted for broader applications in financial auditing, e-procurement, expense management, and other digital transaction verification systems.

Future research directions include: (1) enhancing OCR accuracy through advanced image preprocessing and domain-specific model training, (2) implementing machine learning-based anomaly detection for automated fraud prevention, (3) exploring integration with mobile applications for improved user experience, (4) scaling the system to handle enterprise-level transaction volumes, and (5) extending the blockchain network to enable cross-organizational transparency in supply chain fuel consumption tracking.

\section*{Acknowledgment}
The authors wish to thank Nusa Mandiri University for supporting this research.

\newpage

\begin{thebibliography}{99}
\bibitem{1} Y. Du et al., "PP-OCR: A Practical Ultra Lightweight OCR System," arXiv preprint arXiv:2009.09941, 2020.
\bibitem{2} PaddlePaddle Authors, "PaddleOCR: Awesome Multilingual OCR Toolkits," GitHub repository, 2021.
\bibitem{3} [Invoice Processing with PaddleOCR reference - to be completed]
\bibitem{4} [Blockchain in Supply Chain reference - to be completed]
\bibitem{5} S. Abubo, "Government Document Management System Based on Blockchain and OCR," International Journal of Information Technology, 2024.
\end{thebibliography}

\newpage

\begin{AuthorsProfiles}
\authoritem{fig/kenny.png}{%
\noindent \textbf{Kenny Aldi} is a student at the Faculty of Computer Science, Nusa Mandiri University, with research interests in artificial intelligence, blockchain technology, and document automation systems.}

\authoritem{fig/kursehi.jpeg}{\noindent \textbf{Kursehi Falgenti} is a faculty member at Nusa Mandiri University with expertise in information systems and software engineering.}

\authoritem{fig/nasir.jpeg}{\noindent \textbf{Nasir Hamzah} is a researcher at Nusa Mandiri University specializing in blockchain technology and distributed systems.}

\authoritem{fig/ragil.jpg}{\noindent \textbf{Ragil Yulianto} is a lecturer at Nusa Mandiri University with research focus on artificial intelligence and machine learning applications.}

\authoritem{fig/firman.jpeg}{\noindent \textbf{Muhammad Firmansyah} is a faculty member at Nusa Mandiri University with expertise in computer vision and intelligent systems.}\relax
\end{AuthorsProfiles}

{\fontsize{9pt}{9pt} \selectfont\noindent \textbf{How to cite this paper:} Kenny Aldi, Kursehi Falgenti, Nasir Hamzah, Ragil Yulianto, Muhammad Firmansyah, ``Design of Automated Fuel Consumption Recording System Based on Optical Character Recognition (OCR) and Blockchain Hyperledger Fabric", International Journal of Information Engineering and Electronic Business(IJIEEB), Vol.14, No.6, pp.1-4, 2022. DOI:10.5815/ijieeb.2022.06.01}
\end{document}