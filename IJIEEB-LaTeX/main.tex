\documentclass{ijieeb}

\usepackage{amsmath,amssymb,multirow,array}

\IjieebSet{
    title = {Paper Title: Preparations of Papers for the Journals of the MECS Press},
    volume = {14},
    issue = {6},
    year = {2022},
    month = {December 8},
    startpage = {1},
    endpage = {4},
    doi = {10.5815/ijieeb.2022.06.01},
    abstract = {
        This document presents the required layout of articles to be submitted for publication in print and electronic versions of the journals of MECS Press. The abstract should be 160 words at least. Do not use any abbreviation and equations in your abstract.
    },
    keywords = {Leave one blank line after the Abstract and write your Keywords or/and Key phrases (5-7 words or/and phrases, separated by comma).},
    received = {Date Month, Year; Revised: Date Month, Year; Accepted: Date Month, Year; Published: Date Month, Year}
}

\author*{Firstname A. Lastname}{Name of Institution/Department, City, Zip Code, Country}{first.author@hostname1.org}{https://orcid.org/0000-xxxx-xxxx-xxxx}

\author{Firstname B. Lastname}{Name of Institution/Department, City, Zip Code, Country}{first.author@hostname2.org}{https://orcid.org/0000-xxxx-xxxx-xxxx}

\author{Firstname C. Lastname}{Name of Institution/Department, City, Zip Code, Country}{first.author@hostname3.org}{https://orcid.org/0000-xxxx-xxxx-xxxx}



\thispagestyle{thefirstpage}
\begin{document}


\maketitle



\section{Introduction}
Your goal is to simulate the usual appearance of papers in a Journal of the MECS Press. We are requesting that you follow these guidelines as closely as possible.

\subsection{Full-Sized Camera-Ready (CR) Copy}
Paper size: prepare your CR paper in full-size format, on A4 paper (210 x 297 mm, 8.27 x 11.69 in).

First page margins: top = 30 mm (1.18 in), bottom, left and right = 20 mm (0.79 in). 

Other pages margins: top = 2.5 mm (0.98 in), bottom, left and right = 20 mm (0.79 in). 

\textit{Type sizes and typefaces: Follow the type sizes specified in Table 1. As an aid in gauging type size, 1 point is about 0.35 mm. The size of the lowercase letter “j” will give the point size. Times New Roman has to be the font for main text. Paper should be single spaced.}

Paragraph indentation: first-line 7.4 mm (0.3 in). For Abstract and Index Terms, no first-line indentation.

Alignment: left- and right-justify. Left-Aligned your table captions, figure captions. Center-justy your tables and figures. Use automatic hyphenation and check spelling. Digitize or paste down figures.

Title: use 24-point Times New Roman font. Its paragraph description should be set so that the line spacing is single with 6-point spacing before and 6-point spacing after. Use two additional line spacings of 10 points before the beginning of the Introduction section, as shown above. A title of article should be the fewest possible words that accurately describe the content of the paper. The title should be succinct and informative. Do not use acronyms or abbreviations in your title. Titles are often used in information-retrieval systems. Avoid writing long formulas with subscripts in the title.

Section headings: should be 11-point, Times New Roman Font, Bold, Left-aligned and numbered with Arabic numerals (1, 2, 3, …, except for Acknowledgement and References), followed by a period, two spaces, and Each word (except for Prepositions, Pronouns) first letter should be capitalized, others lowercase. The paragraph description of the  
section heading line should be set for 12 points before and 12 points after. The section or subsection headings should be typed on a separate line, e.g., 1. Introduction.




\loadmaingeometry

\begin{table}
    \centering
    \caption{Type Sizes for Camera-Ready Papers}
    \resizebox{.75\textwidth}{!}{
    \begin{tabular}{|m{.12\textwidth}<{\centering}|m{.5\textwidth}<{\raggedright}|m{.12\textwidth}<{\centering}|m{.12\textwidth}<{\centering}|}
        \hline
        \multirow{2}{*}{Type size (pts.)} & \multicolumn{3}{c|}{Appearance} \\\cline{2-4}
        & Regular &	Bold &	Italic \\\hline
        6 &	Table captions, a table superscripts & & \\\hline
        8 & Section titles, tables, table names, first letters in table captions, figure captions, footnotes, text subscripts, and superscripts & & \\\hline
        9 &	References, authors' biographies & & \\\hline
        10 & Authors' affiliations, main text, equations, first letters in section titles & &	Subheading \\\hline
        11 & & Authors' names Section headings & \\\hline	
        24 & Paper title & & \\\hline	
    \end{tabular}}
\end{table}


\subsection{PDF Creation}
The PDF document should be sent as an open file, i.e. without any data protection. 

Please do not use the Adobe Acrobat PDFWriter to generate the PDF file. Use the Adobe Acrobat Distiller instead, which is contained in the same package as the Acrobat PDFWriter. 

Make sure that you have used Type 1 or True Type Fonts (check with the Acrobat Reader or Acrobat Writer by clicking on File $>$ Document Properties $>$ Fonts to see the list of fonts and their type used in the PDF document). 

As always with a conversion to PDF, authors should very carefully check a printed copy. 


\section{Helpful Hints}
\subsection{Figures and Tables}
Position figures and tables at the center of the page. Figure captions should be Left-Aligned below the figures; table captions should be Left-Aligned above. Avoid placing figures and tables before their first mention in the text. Use the abbreviation “Fig. 1,” even at the beginning of a sentence. 
\begin{figure}[h]
    \centering
    \includegraphics[width = .5\textwidth]{fig/fig1.png}
    \caption{Note how the caption is centered in the column.}
\end{figure}

To figure axis labels, use words rather than symbols. Do not label axes only with units. Do not label axes with a ratio of quantities and units. Figure labels should be legible, about 9-point type.
Color figures will be appearing only in online publication. All figures will be black and white graphs in print publication. 

\subsection{References}
Number citations consecutively in square brackets [1]. No punctuation follows the bracket [2]. Use “Ref. [3]” or “Reference [3]” at the beginning of a sentence: 

Give all authors’ names; use “et al.” if there are six authors or more. Papers that have not been published, even if they have been submitted for publication, should be cited as “unpublished” [4]. Papers that have been accepted for publication should be cited as “in press” [5]. In a paper title, capitalize the first word and all other words except for conjunctions, prepositions less than seven letters, and prepositional phrases.

For papers published in translated journals, first give the English citation, then the original foreign-language citation [6].

For on-line references a URL and time accessed must be given. 

At the end of each reference, give the DOI (Digital Object Identifier) number as long as available, in the format as “doi:10.1518/hfes.2006.27224”

\subsection{Footnotes}
Number footnotes separately in superscripts $^{1, 2, \ldots}$ Place the actual footnote at the bottom of the column in which it was cited, as in this column. See first page footnote for an example. 

Dates of manuscript submission, revision and acceptance should be included in the first page footnote. Remove the first page footnote if you don't have any information there.

\subsection{Abbreviations and Acronyms}
Define abbreviations and acronyms the first time they are used in the text, even after they have been defined in the abstract. Do not use abbreviations in the title unless they are unavoidable.

\subsection{Equations}
Equations should be centered in the column. The paragraph description of the line containing the equation should be set for one-line spacings before and after. Number equations consecutively with equation numbers in parentheses flush with the right margin, as in (1). Italicize Roman symbols for quantities and variables, but not Greek symbols. Punctuate equations with commas or periods when they are part of a sentence, as in
\begin{align}
    a + b = c
\end{align}

Symbols in your equation should be defined before the equation appears or immediately following. Use “(1),” not “Eq. (1)” or “equation (1),” except at the beginning of a sentence: “Equation (1) is ...”

\subsection{Other Recommendations}
Use either SI (MKS) or CGS as primary units. (SI units are encouraged.) If your native language is not English, try to get a native English-speaking colleague to proofread your paper. Do not add page numbers.

\subsection{A Quick Checklist}
\begin{itemize}
    \item \textbf{Paper size}=A4; \textbf{Fisrt Page Margins}: top=3 cm, bottom=left=right=2 cm; \textbf{Other Pages Margins}: top=2.5 cm, bottom=left=right=2 cm.
    \item For the whole document (“Ctrl-A” to select the whole document), \textbf{Font Type}=Times New Roman, do \textbf{NOT} use any Asian font type like SimSun in formulas, section numbers (1, 2, 3, ...), list numbers (1), 2), (1), (2), ...), or punctuation marks (“,”, “.”, “:”, “;”, “(“, “)”, ...). Check Word Count (on the status bar at the bottom of the window) to ensure the number of Asian Characters (including textboxes and footnotes) is 0. 
    \item In Paragraph settings for the whole document (“Ctrl-A” to select the whole document), \textbf{Line spacing} must be “Single”, “Snap to grid” must \textbf{NOT} be checked. 
    \item In Paragraph settings for main text except section titles, \textbf{Indentation} left=right=0, first line=0.74 cm; \textbf{Spacing} before=after=0, not blank line between paragraphs.
    \item \textbf{Title and authors}: font style=regular NOT bold NOT italic; font size for title is 24 point, with 6 spacing before \& after, for authors names font size is 11, bold, affiliations font size is 10.
    \item \textbf{References}: strictly follow the instructions in Section 3.2.
    \item \textbf{Biographies}: it is strongly recommended adding for each author a short bio to the end of the paper.
\end{itemize}


\section*{Appendix A Appendix Title}
Appendixes, if needed, are numbered by A, B, C... Use two spaces before APPENDIX TITLE.

\section*{Acknowledgment}
The authors wish to thank A, B, C. This work was supported in part by a grant from XYZ.


% \nocite{*}
% \bibliographystyle{unsrt}
% \bibliography{ref}

\begin{thebibliography}{99}
\addtolength{\itemsep}{-.6em}
\bibitem{1}	G. Eason, B. Noble, and I. N. Sneddon, “On certain integrals of Lipschitz-Hankel type involving products of Bessel functions,” {\em Phil. Trans. Roy. Soc. London}, vol. A247, pp. 529-551, April 1955. 
\bibitem{2}	J. Clerk Maxwell, {\em A Treatise on Electricity and Magnetism}, 3$^{\rm rd}$ ed., vol. 2. Oxford: Clarendon, 1892, pp.68-73.
\bibitem{3}	I. S. Jacobs and C. P. Bean, “Fine particles, thin films and exchange anisotropy,” in {\em Magnetism}, vol. III, G. T. Rado and H. Suhl, Eds. New York: Academic, 1963, pp. 271-350.
\bibitem{4}	K. Elissa, “Title of paper if known,” unpublished.
\bibitem{5}	R. Nicole, “Title of paper with only first word capitalized”, {\em J. Name Stand. Abbrev}., in press.
\bibitem{6}	Y. Yorozu, M. Hirano, K. Oka, and Y. Tagawa, “Electron spectroscopy studies on magneto-optical media and plastic substrate interface,” {\em IEEE Transl. J. Magn. Japan}, vol. 2, pp. 740-741, August 1987 [Digests 9\textsuperscript{th} Annual Conf. Magnetics Japan, p. 301, 1982].
\bibitem{7}	M. Young, {\em The Technical Writer's Handbook}. Mill Valley, CA: University Science, 1989.
\end{thebibliography}

\vspace*{1cm}
\begin{AuthorsProfiles}
\authoritem{fig/author.png}{%
\noindent \textbf{Firstname A. Lastname},  and the other authors may include biographies and photographs at the end of regular papers. Photographs, if provided, should be cropped into 26mm in width and 32mm in height. The first paragraph may contain a place and/or date of birth (list place, then date). Next, the author's educational background is listed. The degrees should be listed with type of degree in what field, which institution, city, state or country, and year degree was earned. The author's major field of study should be lower-cased.

The second paragraph uses the pronoun of the person (he or she) and not the author's last name. It lists military and work experience, including summer and fellowship jobs. Job titles are capitalized. The current job must have a location; previous positions may be listed without one. Information concerning previous publications may be included. Try not to list more than three books or published articles. The format for listing publishers of a book within the biography is: title of book (city, state: publisher name, year) similar to a reference. Current and previous research interests end the paragraph.

The third paragraph begins with the author's title and last name (e.g., Dr. Smith, Prof. Jones, Mr. Kajor, Ms. Hunter). List any memberships in professional societies like the IEEE. Finally, list any awards and work for professional committees and publications. Personal hobbies should not be included in the biography.}

\authoritem{fig/author.png}{\noindent \textbf{Firstname B. Lastname} includes the biography here.}

\authoritem{fig/author.png}{\noindent \textbf{Firstname C. Lastname} includes the biography here.}\relax
\end{AuthorsProfiles}


{\fontsize{9pt}{9pt} \selectfont\noindent \textbf{How to cite this paper:} Firstname A. Lastname, Firstname B. Lastname, Firstname C. Lastname, ``Paper Title: Preparations of Papers for the Journals of the MECS Press", International Journal of Information Engineering and Electronic Business(IJIEEB), Vol.14, No.6, pp.1-4, 2022. DOI:10.5815/ijieeb.2022.06.01}
\end{document}